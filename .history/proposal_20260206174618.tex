% CVPR 2022 Paper Template
% based on the CVPR template provided by Ming-Ming Cheng (https://github.com/MCG-NKU/CVPR_Template)
% modified and extended by Stefan Roth (stefan.roth@NOSPAMtu-darmstadt.de)

\documentclass[10pt,twocolumn,letterpaper]{article}

%%%%%%%%% PAPER TYPE  - PLEASE UPDATE FOR FINAL VERSION
%\usepackage[review]{cvpr}      % To produce the REVIEW version
\usepackage{cvpr}              % To produce the CAMERA-READY version
%\usepackage[pagenumbers]{cvpr} % To force page numbers, e.g. for an arXiv version
% Include other packages here, before hyperref.
\usepackage{graphicx}
\usepackage{amsmath}
\usepackage{amssymb}
\usepackage{booktabs}
\usepackage{float}


% It is strongly recommended to use hyperref, especially for the review version.
% hyperref with option pagebackref eases the reviewers' job.
% Please disable hyperref *only* if you encounter grave issues, e.g. with the
% file validation for the camera-ready version.
%
% If you comment hyperref and then uncomment it, you should delete
% ReviewTempalte.aux before re-running LaTeX.
% (Or just hit 'q' on the first LaTeX run, let it finish, and you
%  should be clear).
\usepackage[pagebackref,breaklinks,colorlinks]{hyperref}


% Support for easy cross-referencing
\usepackage[capitalize]{cleveref}
\crefname{section}{Sec.}{Secs.}
\Crefname{section}{Section}{Sections}
\Crefname{table}{Table}{Tables}
\crefname{table}{Tab.}{Tabs.}


%%%%%%%%% PAPER ID  - PLEASE UPDATE
\def\cvprPaperID{*****} % *** Enter the CVPR Paper ID here
\def\confName{CVPR}
\def\confYear{2022}


\title{COMP472 Proposal (subject to change)} \author{ Baila Ly {\tt\small 4027963} \quad Sanjay Thambithrurai {\tt\small 4018440} \quad Benjamin Zitella {\tt\small 4021138} \quad Jia Hao To {\tt\small 40263401} \quad Rasel Abdul Samad {\tt\small 4020992} } 

\begin{document}
	\maketitle
	%%%%%%%%% BODY TEXT
	\section{Problem Statement}
	\label{sec:intro}
	Identifying footwear from social media imagery (Instagram, Pinterest) is challenging due to high intra-class variance,
	inconsistent lighting, and obscured branding. This project develops a hierarchical shoe classification system using
	Convolutional Neural Networks (CNNs) to bridge the gap between user-uploaded photos and specific product
	metadata. Our goal is to transition from broad category recognition (e.g., Casual vs. Dress) to granular brand and
	model identification.
	[text to change]
	
	\section{Dataset Selection}
	\label{sec:dataset}
	\begin{table}[H]
		\centering
		\begin{tabular}{@{}l c @{\hspace{1cm}} c c@{}}
			\toprule
			Dataset & Class & Imag/Class & Resolution\\
			\midrule
			name1 & 0 & 0 & 0\\
			name2 & 0 & 0 & 0\\
			name3 & 0 & 0 & 0\\
			\bottomrule
		\end{tabular}
		\caption{insert text}
		\label{tab:example}
	\end{table}
	
	\section{Methodology}
	\label{sec:Method}
		\subsection{Pipeline}
	Our pipeline focuses a transfer learning approach to handle the diverse resolutions and backgrounds of the aggregated
	data.All images will be standardized to 224 x 224 pixels and normalized. To prevent overfitting on smaller
	brand-specific sets,we will apply real-time data augmentation (flips, rotations, and brightness shifts). We will
	benchmark a custom baseline CNN against a pre-trained ResNet-50 architecture to leverage its residual learning
	capabilities for complex feature extraction. [To be changed]
	
	\pagebreak
	\section{Gantt Chart}
	\label{sec:chart}
	
	
	\pagebreak
	\section{Bibliography}
	
	
	%%%%%%%%% REFERENCES
	{\small
		\bibliographystyle{ieee_fullname}
		\bibliography{egbib}
	}
	
\end{document}
